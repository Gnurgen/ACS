\documentclass{article}
\usepackage{times}
\usepackage{balance}
\usepackage{amssymb}
\usepackage{amsfonts}
\usepackage{amsmath}
\usepackage[amsmath,thmmarks]{ntheorem}
\usepackage{mathrsfs}
\usepackage[utf8]{inputenc}
\usepackage{listings}              % code insert
\usepackage{graphicx}

\newcommand{\R}{\mathbb{R}}
\newcommand{\C}{\mathbb{C}}
\newcommand{\Z}{\mathbb{Z}}
\newcommand{\id}{\textrm{id}}
\newcommand{\pr}{\mathrm{pr}}
\newcommand{\N}{\mathbb{N}}


\newtheorem{thm}[equation]{Theorem}
\newtheorem{lem}[equation]{Lemma}
\newtheorem{prop}[equation]{Proposition}
\newtheorem{cor}[equation]{Corollary}
\newtheorem{conj}[equation]{Conjecture}

\theoremstyle{plain}
\theorembodyfont{\normalfont}
\newtheorem{defn}[equation]{Definition}
\newtheorem{ex}[equation]{Example}
\newtheorem{claim}[equation]{Claim}

\theoremstyle{nonumberplain}
\theoremheaderfont{\normalfont\bfseries}
\theorembodyfont{\normalfont}
\theoremsymbol{\ensuremath{\square}}
\theoremseparator{.}
\newtheorem{proof}{Proof}


\pagestyle{plain}

\begin{document}

\title{Advanced Computer Systems \\ Assignment 4}

\author{Anna Sofie Kiehn and Kenneth Jürgensen}

\maketitle

\section*{Question 1: Recovery Concepts}

\subsection*{1}
	Since force is implemented there is no need for undo as all changes are written when comitted. Since there is no-steal no data is written before it is comitted, and therefore no undo scheme needs to be implemented as there can be no data that is written but not comitted.

\subsection*{2}
	Nonvolatile storage is storage that does not lose data when power is lost, like a harddisk. Stable storage is storage that which in theory can survive any kind of failure. Nonvolatile storage,while it does not lose data on power failure it is still succeptible to data corruption due to partially written data, where as stable storage can use replication of data to rule out corruption. However this replication of data also makes stable storage more computationally expensive than nonvolatile storage.

\subsection*{3}
	When the change to the database is to be written to disk, the write-ahead-log must be written to stable storage before the change is written. This is to ensure that no change of the state of the database can happen without being recorded for potential redo/undo in case of a crash.

\section*{Question 2: ARIES}

\subsection*{1}

Below is the state of the transaction and dirty page tables after the analysis phase.
The U in the transaction table, referes to the transactions needs to be undone.\\
\begin{tabular}{| c | c | c |}
	\hline
	\multicolumn{3}{|c|}{Transaction table} \\
	\hline
	transID & lastLSN & Status\\
	\hline
	T1 & 4 & U\\
	\hline
	T2 & 9 & U\\
	\hline
\end{tabular}
\begin{tabular}{| c | c |}
	\hline
	\multicolumn{2}{|c|}{Dirty page table} \\
	\hline
	pageID & recLSN \\
	\hline
	P2 & 3 \\
	\hline
	P1 & 4 \\
	\hline
	P5 & 5 \\
	\hline
	P3 & 6 \\
	\hline
\end{tabular}

\subsection*{2}

As we can in the transaction table above the set of winner transactions are \{T3\}, as this T3 finishes before the crash, the set of losers are \{T1,T2\}, as they are both active when the crash happens.

\subsection*{3}
Redo phase starts at LSN 3 .\\
Undo phase ends at LSN 3.

\subsection*{4}
The set of log records that may cause page redos are \{3,4,5,6,8,9\}.

\subsection*{5}
The set of records that are undone is \{9,8,5,4,3\}.

\subsection*{6}
\begin{tabular}{ l  l  l  l  l  l }
	LSN & LAST\_LSN & TRAN\_ID & TYPE & PAGE\_ID & undoNextLSN \\
	- - - & - - - - - - - - & - - - - - - - & - - - - & - - - - - - - & - - - - - - - - - - - \\
	1 & - & - & begin CKPT & - & - \\
	2 & - & - & end CKPT & - & - \\
	3 & NULL & T1 & update & P2 & - \\
	4 & 3 & T1 & update & P1 & - \\
	5 & NULL & T2 & update & P5 & - \\
	6 & NULL & T3 & update & P3 & - \\
	7 & 6 & T3 & commit & - & - \\
	8 & 5 & T2 & update & P5 & - \\
	9 & 8 & T2 & update & P3 & - \\
	10 & 6 & T3 & end & - & - \\
	11 & - & T2 & CLR & P3 & 8 \\
	12 & - & T2 & CLR & P5 & 5 \\
	13 & - & T2 & CLR & P5 & NULL \\
	14 & - & T1 & CLR & P1 & 3 \\
	15 & - & T1 & CLR & P2 & NULL \\
\end{tabular}

\section{Programming task}

\subsection{Setup for experiments}

\subsection{Plots of throughput and latency}

\subsection{Liability of the matrics}

\end{document}