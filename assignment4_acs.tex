\documentclass{article}
\usepackage{times}
\usepackage{balance}
\usepackage{amssymb}
\usepackage{amsfonts}
\usepackage{amsmath}
\usepackage[amsmath,thmmarks]{ntheorem}
\usepackage{mathrsfs}
\usepackage[utf8]{inputenc}
\usepackage{listings}              % code insert
\usepackage{graphicx}
\usepackage{rotating}

\newcommand{\R}{\mathbb{R}}
\newcommand{\C}{\mathbb{C}}
\newcommand{\Z}{\mathbb{Z}}
\newcommand{\id}{\textrm{id}}
\newcommand{\pr}{\mathrm{pr}}
\newcommand{\N}{\mathbb{N}}


\newtheorem{thm}[equation]{Theorem}
\newtheorem{lem}[equation]{Lemma}
\newtheorem{prop}[equation]{Proposition}
\newtheorem{cor}[equation]{Corollary}
\newtheorem{conj}[equation]{Conjecture}

\theoremstyle{plain}
\theorembodyfont{\normalfont}
\newtheorem{defn}[equation]{Definition}
\newtheorem{ex}[equation]{Example}
\newtheorem{claim}[equation]{Claim}

\theoremstyle{nonumberplain}
\theoremheaderfont{\normalfont\bfseries}
\theorembodyfont{\normalfont}
\theoremsymbol{\ensuremath{\square}}
\theoremseparator{.}
\newtheorem{proof}{Proof}


\pagestyle{plain}

\begin{document}

\title{Advanced Computer Systems \\ Assignment 4}

\author{Anna Sofie Kiehn and Kenneth Jürgensen}

\maketitle

\section*{Question 1: Recovery Concepts}

\subsection*{1}
	Since force is implemented there is no need for undo as all changes are written when comitted. Since there is no-steal no data is written before it is comitted, and therefore no undo scheme needs to be implemented as there can be no data that is written but not comitted.

\subsection*{2}
	Nonvolatile storage is storage that does not lose data when power is lost, like a harddisk. Stable storage is storage that which in theory can survive any kind of failure. Nonvolatile storage,while it does not lose data on power failure it is still succeptible to data corruption due to partially written data, where as stable storage can use replication of data to rule out corruption. However this replication of data also makes stable storage more computationally expensive than nonvolatile storage.

\subsection*{3}
	When the change to the database is to be written to disk, the write-ahead-log must be written to stable storage before the change is written. This is to ensure that no change of the state of the database can happen without being recorded for potential redo/undo in case of a crash.

\section*{Question 2: ARIES}

\subsection*{1}

Below is the state of the transaction and dirty page tables after the analysis phase.
The U in the transaction table, referes to the transactions needs to be undone.\\
\begin{tabular}{| c | c | c |}
	\hline
	\multicolumn{3}{|c|}{Transaction table} \\
	\hline
	transID & lastLSN & Status\\
	\hline
	T1 & 4 & U\\
	\hline
	T2 & 9 & U\\
	\hline
\end{tabular}
\begin{tabular}{| c | c |}
	\hline
	\multicolumn{2}{|c|}{Dirty page table} \\
	\hline
	pageID & recLSN \\
	\hline
	P2 & 3 \\
	\hline
	P1 & 4 \\
	\hline
	P5 & 5 \\
	\hline
	P3 & 6 \\
	\hline
\end{tabular}

\subsection*{2}

As we can see in the transaction table above the set of winner transactions are \{T3\}, as this T3 finishes before the crash, the set of losers are \{T1,T2\}, as they are both active when the crash happens.


\subsection*{3}
Redo phase starts at LSN 3 as that is the smallest recLSN in the dirty page table.\\
Undo phase ends at LSN 3 as that is the oldest log record of the transactions active at the crash.

\subsection*{4}
The set of log records that may cause page redos are \{3,4,5,6,8,9\}.

\subsection*{5}
The set of records that are undone is \{9,8,5,4,3\}. The undo phase starts at LSN 9 and undos all updates done by the looser transactions. 

\subsection*{6}
\begin{tabular}{ l  l  l  l  l  l }
	LSN & LAST\_LSN & TRAN\_ID & TYPE & PAGE\_ID & undoNextLSN \\
	- - - & - - - - - - - - & - - - - - - - & - - - - & - - - - - - - & - - - - - - - - - - - \\
	1 & - & - & begin CKPT & - & - \\
	2 & - & - & end CKPT & - & - \\
	3 & NULL & T1 & update & P2 & - \\
	4 & 3 & T1 & update & P1 & - \\
	5 & NULL & T2 & update & P5 & - \\
	6 & NULL & T3 & update & P3 & - \\
	7 & 6 & T3 & commit & - & - \\
	8 & 5 & T2 & update & P5 & - \\
	9 & 8 & T2 & update & P3 & - \\
	10 & 6 & T3 & end & - & - \\
	11 & - & T2 & CLR & P3 & 8 \\
	12 & - & T2 & CLR & P5 & 5 \\
	13 & - & T2 & CLR & P5 & NULL \\
	14 & - & T1 & CLR & P1 & 3 \\
	15 & - & T1 & CLR & P2 & NULL \\
\end{tabular}
We were a little unclear about if the records that are undone are removed from the log or if they are kept there, but the CLR record indicates that the record has been rolled back. We think that a log should never delete a record, once something has been logged that information should not be lost, so we show the log with the CLR to indicate that the records have been undone.

\section{Programming task}


\subsection{Setup for experiments}

\textbf{Generation of data:}
We implemented the random generation of books by rolling a random number between 0 and 1 billion as the isbn, to make it very unlikely that two threads wil generate two different books with the same isbn. The number of copies for each book is a random number between 1000 and 2000, the price is a random number between 100 and 600 and the title and author are random 10 character strings consisting of lower case letters and space. Every generated book has 0 in saleMisses, timesRated and totalRating and a random boolean in editorPick. The bookstore is always initiated with 20 books generated this way.\\\\
\textbf{Hardware:}
We ran the tests on a macbook with the following specs:
\begin{itemize}
	\item OS: OS X Yosemite version 10.10.5
	\item RAM: 8 Gb 1600 MHz DDR3
	\item CPU: 2 GHz Intel Core i7
\end{itemize}
\textbf{Workload configuration:}
We configured every worker to get up to 10 editorpicks and chose 5 books from this set to by 1 copy of each book as the client interaction which is run roughly 60\% of the total runs. For the replenishment interaction we chose to add 10 copies to the 5 books with the least copies, which is run about 30\% of the total runs. Finally for the stock acquisition interaction we chose to add 5 new books which is run roughly 10\% of the total runs.


\subsection{Plots of throughput and latency}

Figure 1 shows the throughput at various amounts of concurrent clients when running the test locally and using RPC and figure 2 shows latency at various amounts of concurrent clients when running the test locally and using RPC.

\centering
\begin{figure}
	\hspace*{-2.2in}
	\includegraphics[scale=0.85]{throughput}
	\caption{Plot of the measurements of throughput}
\end{figure}
\begin{figure}
	\hspace*{-2.2in}
	\includegraphics[scale=0.85]{latency}
	\caption{Plot of the measurements of latency}
\end{figure}

\subsection{Reliability of the metrics}

\end{document}